\documentclass[12pt]{article}        % с помощью documentclass выбираем тип документа. В зависимости от него будет разное форматирование
\usepackage[utf8]{inputenc}          % этот устанавливает кодировку документа
\usepackage[T2A]{fontenc}            % T2A - 8-битная кодировка, используемая с кириллицей
\usepackage[english, russian]{babel} % этот пакет нужен для работы с русским языком
\usepackage{color}

% всё что вводиться до начала команды \begin{document} считается преамбулой. В ней можно определить тип создаваемого документа, используемый язык, нужные библиотеки и ряд других элементов
%-------------------------------------------------------------------------------------
\title{Изучение \LaTeX} % заголовок документа
\author{Сергей Маслов}  % указание автора и сноска
\date{2023.01.03}       % указание времени в заголовке
%-------------------------------------------------------------------------------------
\begin{document}
\maketitle % эта команда отвечает за отображение информации переданной в нее в преамбуле

Команды для форматирования текста:
\begin{itemize} % неупорядоченный список
    \item Жирный текст \textbf{\\textbf\{...\}}
    \item Курсивный текст \textit{\\textit\{...\}}
    \item Нижнее подчеркивание \underline{\\underline\{...\}}
\end{itemize}

Использованные сайты:
\begin{enumerate} % упорядоченный список
    \item https://grammarware.net/text/syutkin/RusInLaTeX.pdf
    \item https://habr.com/ru/company/ruvds/blog/574352/
\end{enumerate}

Одно из главных удобств \LaTeX состоит в простоте использования математических выражений.
Этот инструмент предоставляет два режима их написания: режим inline (встраивание) и режим display (отображение). Первый используется для написания формул, являющихся частью текста. Второй позволяет создавать выражения, не входящие в состав текста или абзаца, а размещаемые на отдельных строках.
\begin{verbatim}
print(hello world)
\end{verbatim}

\end{document}
