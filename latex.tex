\documentclass[12pt]{article}        % с помощью documentclass выбираем тип документа. В зависимости от него будет разное форматирование
\usepackage[utf8]{inputenc}          % этот устанавливает кодировку документа
\usepackage[T2A]{fontenc}            % T2A - 8-битная кодировка, используемая с кириллицей
\usepackage[english, russian]{babel} % этот пакет нужен для работы с русским языком
\usepackage{color}
\usepackage{graphicx}

% всё что вводиться до начала команды \begin{document} считается преамбулой. В ней можно определить тип создаваемого документа, используемый язык, нужные библиотеки и ряд других элементов
%-------------------------------------------------------------------------------------
\title{Изучение \LaTeX} % заголовок документа
\author{Сергей Маслов}  % указание автора и сноска
\date{2023.01.03}       % указание времени в заголовке
%-------------------------------------------------------------------------------------
\begin{document}
\maketitle % эта команда отвечает за отображение информации переданной в нее в преамбуле

\begin{abstract}
В этом документе разбирются возможности изыка \LaTeX
\end{abstract}

\tableofcontents

\subsection{Форматироване текста и списки}
Команды для форматирования текста:
\begin{itemize} % неупорядоченный список
    \item Жирный текст \textbf{\\textbf\{...\}}
    \item Курсивный текст \textit{\\textit\{...\}}
    \item Нижнее подчеркивание \underline{\\underline\{...\}}
\end{itemize}

Использованные сайты:
\begin{enumerate} % упорядоченный список
    \item https://grammarware.net/text/syutkin/RusInLaTeX.pdf
    \item https://habr.com/ru/company/ruvds/blog/574352/
\end{enumerate}

\subsection{Код}
\begin{verbatim}
# здесь есть куски которые форматируются как код
g = 'Hello World'
print(g)
\end{verbatim}

\subsection{Математические формулы}
Одно из главных удобств \LaTeX состоит в простоте использования математических выражений.
Этот инструмент предоставляет два режима их написания: режим inline (встраивание) и режим display (отображение). Первый используется для написания формул, являющихся частью текста. Второй позволяет создавать выражения, не входящие в состав текста или абзаца, а размещаемые на отдельных строках.


\underline{Вот пример встраиваемой формулы.} Теорема пифагора для прямоугольного треугольника \textbf{ABC} \begin{math}AC^2 = AB^2 + BC^2\end{math}, где $AC^2$ - это гиппотенуза, а $AB^2$ и $BC^2$ катеты.


\underline{A вот пример \textit{дисплейной формулы}:}
\begin{displaymath} G =  \frac{mc^2}{2_i}\end{displaymath}

\subsection{Изображения}
\begin{figure}[h]
    \centering
    \includegraphics[width=0.5\textwidth]{image_1}
    \caption{Структурирование документа}
    \label{fig:image1}
\end{figure}

\ref{fig:image1}: этот код будет замещен числом, соответствующим изображению, на которое делается ссылка.
\pageref{fig:image1} - это страница на которой есть иззображение

\subsection*{Таблицы}
\addcontentsline{toc}{subsection}{Таблицы}

\begin{center}
\begin{tabular}{ c c c }
 cell1 & cell2 & cell3 \\ 
 cell4 & cell5 & cell6 \\  
 cell7 & cell8 & cell9    
\end{tabular}
\end{center}

По умолчанию для создания таблиц в LaTeX используется окружение tabular. В этом окружении нужно указывать параметр, в нашем случае {c c c}. В таком виде он сообщит LaTeX, что в таблице будет три столбца, и текст внутри этих столбцов нужно разместить по центру. Можно также использовать r для выравнивания текста по правому краю и l для выравнивания по левому.

Для указания разрывов в записях таблицы используется символ \&. Чтобы перейти к очередной строке таблицы используется команда создания новой строки, два обратных слеша. Всю таблицу мы заключаем в окружение center, чтобы она размещалась по центру страницы.

\begin{center}
 \begin{tabular}{||c c | c c||} 
 \hline
 Col1 & Col2 & Col2 & Col3 \\ [0.5ex] 
 \hline\hline
 1 & 6 & 87837 & 787 \\ 
 \hline
 2 & 7 & 78 & 5415 \\
 \hline
 3 & 545 & 778 & 7507 \\
 \hline
 4 & 545 & 18744 & 7560 \\
 \hline
 5 & 88 & 788 & 6344 \\ [1ex] 
 \hline
\end{tabular}
\end{center}


Как и изображения, таблицы можно пописывать и делать на них ссылки. Единственное отличие – это использование вместо figure окружения table.\\
Table \ref{table:data} is an example of referenced \LaTeX{} elements.

\begin{table}[h!]
\centering
\begin{tabular}{|c|c|c|c|} 
 \hline
 Col1 & Col2 & Col2 & Col3 \\ [0.5ex] 
 \hline
 1 & 6 & 87837 & 787 \\ 
 2 & 7 & 78 & 5415 \\
 3 & 545 & 778 & 7507 \\
 4 & 545 & 18744 & 7560 \\
 5 & 88 & 788 & 6344 \\ [1ex] 
 \hline
\end{tabular}
\caption{Table to test captions and labels}
\label{table:data}
\end{table}
\end{document}
